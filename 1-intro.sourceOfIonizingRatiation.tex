Though it is possible to generate ionizing radiation through machine sources, the scope of this thesis will only include unstable nuclei and atoms. If there is an energy difference between configurations of a nucleus, there exists the possibility for the nucleus to transform into a more stable configuration and emit radiation in the processes. This may be in the form of particles, such as heavy ions or electrons, or in the form of electromagnetic radiation. The heavy ions are typically either alpha-rays or fragments from fission reactions. The light ions include electrons and their anti-matter equivalent, positrons. The electromagnetic radiation includes gamma-rays, characteristic x-rays, bremsstrahlung x-rays, and annihilation radiation, \cite{johnson_health_physics_intro_5ed}. 


\subsection{Alpha-Decay and Spontaneous Fission}\label{toc:intro.ss.alphaAndFission}



\subsection{Isobaric Decays}\label{intro.toc:ss.isobaricDecays}

\subsection{Isomeric Transitions}

\subsection{Atomic Transitions}

\subsection{Mathematical Description of Radioactive Decay}
